\documentclass[fontsize=10pt,paper=a4,open=any,
twoside=no,toc=listof,toc=bibliography,headings=optiontohead,
captions=nooneline,captions=tableabove,english,DIV=15,numbers=noenddot,final,parskip=half-,
headinclude=true,footinclude=false,BCOR=0mm]{scrartcl}
\pdfvariable suppressoptionalinfo 512\relax
\synctex=1

\usepackage{hirostyle}
\usepackage{hiromacros}
\addbibresource{references.bib}

\title{Report on the Reservoir Engineering Efforts}
\date{2023}

\begin{document}
\maketitle
\tableofcontents

\section{Equations of Motion for a Modulated Fiber Loop}
\label{sec:equat-moti-modul}
\subsection{Introduction}
\label{sec:introduction}
To obtain an equation of motion for the electric field that can be
interpreted as a Hamiltonian, we have to reduce the wave equation to
first order in time. Here we work with the paraxial approximation,
ignoring any transverse fields. A more rigorous treatment, to be used
in numeric simulations, can be derived from wave guide theory as
in~\refcite{Yuan2018a,Haus1984}.

We ultimately want to treat a ring modulator with a space and time
dependent refractive index \(n = n_{0} + n_{1}(\vb{r}, t)\) with
\(n_{1}\ll n_{0}\). This situation is close to the case where
\(n_{1}=0\) where the wave equation can be solved by a plane-wave
ansatz.

To capture near-adiabatic deviations from these solutions, we split off
the fast time evolution of the electric field
\begin{equation}
  \label{eq:1}
  \vb{E}(\vb{r}, t) = \vb{E}_{0}(\vb{r}, t) \eu^{-\i ω t},
\end{equation}
where \(ω\) is as yet undetermined. Now, we \emph{assume} that
\(\dot{\vb{E}}_{0}\sim Ω \cdot\vb{E}_{0}\) with a characteristic
frequency \(Ω \ll ω\). This assumption will have to be verified in the
final result to guarantee consistency. We define the small parameter
\(δ = \frac{Ω}{ω} \ll 1\) for convenience.

For our purposes the magnetic permeability is constant \(μ=μ_{0}\),
whereas the permittivity \(ε(\vb{r}, t)\) is time dependent with
\(\dot{ε} \sim Ω ε\). As we are not taking spatial derivatives of
\(ε\), the spatial argument will be suppressed in the following.

The electric field is real valued, although we do not explicitly take
the real part for now.

\subsection{A Perturbative Maxwell Equation for a Slowy Changing
  Envelope}
\label{sec:pert-maxw-equat}
Applying a second curl to the Maxwell equation\footnote{Which is the
  canonical way to derive the wave equation.} \(\nabla \times\vb{E} =
- ∂_{t}\vb{{B}}\) leads to
\begin{equation}
  \label{eq:2}
  \begin{aligned}
    \nabla \times (\nabla \times \vb{E}) &= -{\nabla}^{2} \vb{E} =
    -∂_{t}\bqty{\mu ∂_{t}\pqty{ε\vb{E}}}=-\mu\pqty{\ddot{ε} \vb{E} + 2
                                           \dot{ε}\dot{\vb{E}} + ε \ddot{\vb{E}}} \\
    &= -μ ω^{2} \bqty{\underbrace{-ε\vb{E}_{0}}_{\sim δ^{0}} + \underbrace{2 \frac{\dot{ε}}{ω}
      \vb{E}_{0} -
      2\iu \frac{ε}{ω}\dot{\vb{E}}_{0}}_{\sim δ^{1}} + \underbrace{2 \frac{\dot{ε}}{ω^{2}}
      \dot{\vb{E}}_{0} + \frac{ε}{ω^{2}}\ddot{\vb{E}}_{0}}_{\sim δ^{2}}}\eu^{-\iu ω t}
  \end{aligned}
\end{equation}

Up to this point we have not made any approximation. We now proceed to
drop the terms of second order in \(δ\).

This leaves us with
\begin{equation}
  \label{eq:3}
  \nabla^{2}\vb{E}_{0} = μ ω^{2} \bqty{\pqty{\frac{2\dot{ε}}{ω} - ε}
    \vb{E}_{0} - \frac{2 \iu ε}{ω} \dot{\vb{E}}_{0}} = \frac{n^{2} ω^{2}}{c^{2}} \bqty{\pqty{\frac{4\dot{n}}{nω} - 1}
    \vb{E}_{0} - \frac{2 \iu }{ω} \dot{\vb{E}}_{0}},
\end{equation}
with \(n=\sqrt{ε μ} c\) which can be rearranged into a form that resembles the
Schr\"odinger equation
\begin{equation}
  \label{eq:4}
  \iu ∂_{t}\vb{E}_{0} =  - \frac{c^{2}}{2 n^{2} ω} \nabla^{2}\vb{E}_{0} +
  \frac{1}{2}\pqty{\frac{4\dot{n}}{n}-ω}
  \vb{E}_{0}.
\end{equation}
Note however, that the ``mass'' \(ωn^{2}/c^{2}\) in the kinetic term is not
constant, giving rise to non hermitian dynamics. This is an artifact
of neglecting orders of \(δ\). We will find however, that in the
situation investigated in \cref{sec:modul-small-port} the violation of
Hermiticity is negligible.

In contrast to the result in \cite{Dutt2019}, \cref{eq:4} is still
second order in space and, provided \(ε\) is real, hermitian.

\subsection{Application to a Ring Resonator}
\label{sec:appl-ring-reson}
As we are describing a fiber ring of radius \(R\) it is convenient to
work in cylindrical coordinates \((ρ, ϕ, z)\). We are interested in
the electric field at the center of the fiber and assume that it does
not vary much in the transversal directions. Thus, we neglect the
dependence of the field on the \(z\) and \(ρ\) directions and make an
ansatz \(\vb{E}_{0} = E(ϕ, t, ρ=R)\hat{\vb{z}}\).

To satisfy the boundary condition \(E(t, 0) = E(t, 2π)\) \(\forall
t\), we expand the field into a Fourier series
\begin{equation}
  \label{eq:6}
  E(ϕ, t) = ∑_{m=-∞}^{+∞} C_{m} a_{m}(t) \eu^{\iu m ϕ},
\end{equation}
with \(C_{m}\) chosen appropriately later, so as to make the \(a_{m}\)
dimensionless. Note that to obtain the \(a_{m}\) of \cite{Dutt2019}
one has to make the substitution \(a_{m}(t) \to a_{-m}(t)\).

In the case of a constant refractive index \(n_{0}\), the modes solve
the ordinary wave equations so that
\(a_{m}\sim \eu^{\iu (mϕ - ω_{m}t)}\) with\footnote{This also deviates
  from \cite{Dutt2019}. There \(m\geq 0\) and also the negative \(ω\)
  solutions are missing. One has to include either one or the other to
  capture all solutions. The reason for this lies their explicit
  construction of a real solution. However, the fact that frequencies
  with a different sign relative to the wave vector exist is not
  captured by their first order differential equation.}
\(ω_{m} = \pm \frac{m c}{R n_{0}}\).  Applying the procedure of
\cref{sec:pert-maxw-equat}, but to each of the mode amplitudes
\(a_{m}=b_{m}\eu^{-\iu ω_{m}t}\) (with \(ω_{m}\) as yet unspecified)
we find
\begin{equation}
  \label{eq:7}
  \begin{aligned}
    \frac{1}{R^{2}} ∑_{m=-∞}^{+∞} C_{m} b_{m}(t)\eu^{-iω_{m}t} ∂_{ϕ}^{2}\eu^{\iu m ϕ}
    &=  \frac{-1}{R^{2}} ∑_{m=-∞}^{+∞} C_{m}m^{2} b_{m}(t) \eu^{\iu (m ϕ-ω_{m}t)}\\
    &= ∑_{m=-∞}^{+∞}C_{m}\frac{n^{2} ω_{m}^{2}}{c^{2}} \bqty{\pqty{\frac{4\dot{n}}{nω_{m}} - 1}
    b_{m}(t) - \frac{2 \iu }{ω_{m}} \dot{b}_{m}(t)}\eu^{\iu (m ϕ-ω_{m}t)}.
  \end{aligned}
\end{equation}

In the limit \(\dot{n}\to 0 \implies n(ϕ,t) = n_{0}=\mathrm{const}\)
we should recover \(\dot{b}_{m}=0\), which implies
\begin{equation}
  \label{eq:8}
  ω_{m}^{2}=\pqty{\frac{mc}{Rn_{0}}}^{2} \implies ω_{m} = \pm
  \abs{\frac{mc}{Rn_{0}}},
\end{equation}
which defines the \(ω_{m}\) in this case. However, to correctly
determine the \(ω_{m}\), a slightly more delicate argument has to be
made.


Evaluating the \(∂_{ϕ}\) derivative, rearranging, applying
\((2π)^{-1}\int_{0}^{2π}\dd{ϕ} \eu^{-\iu l ϕ}\) and defining \(n(ϕ,t)
= n_{0} + n_{1}(ϕ, t)\) yields
\begin{equation}
  \label{eq:9}
  \dot{b}_{l}=-\iu ∑_{m}\bqty{κ_{lm} + γ_{lm}}\eu^{-\iu (ω_{m}-ω_{l}) t}b_{m},
\end{equation}
with
\begin{align}
  κ_{lm}&= \frac{C_{m}}{4π
          ω_{l}C_{l}}∫_{0}^{2π}\pqty{\frac{m^{2}c^{2}}{n^{2}R^{2}} - ω_{m}^{2}}\eu^{\iu(m-l) ϕ}\dd{ϕ}  \overset{\cref{eq:8}}{=}
          A_{lm} ∫_{0}^{2π}\pqty{\frac{n_{0}^{2}}{n^{2}} - 1}\eu^{\iu(m-l)ϕ}\dd{ϕ}  \label{eq:10}\\
  \label{eq:11}
  γ_{lm}&= A_{lm}∫_{0}^{2π}\pqty{\frac{4\dot{n}(ϕ, t)}{ω_{m}n(ϕ,
          t)}}\eu^{\iu(m-l) ϕ}\dd{ϕ}\\
  \label{eq:12}
  A_{lm}&=\frac{C_{m}ω_{m}^{2}}{4π
          ω_{l}C_{l}}=\frac{1}{4π}\frac{C_{m}}{C_{l}}\frac{m^{2}}{l}\frac{c}{Rn_{0}} =\frac{1}{4π}\frac{C_{m}}{C_{l}}\frac{m^{2}}{l}Ω_{R}
\end{align}
which is a Sch\"odinger equation with the Hamiltonian
\(H_{lm}=\bqty{κ_{lm} + γ_{lm}}\). The denominator of \cref{eq:12} may
be cause for concern in the case that \(l=0\). This would imply
\(ω_{l}=0\) which breaks our assumption \(δ\ll 1\). The sum over \(m\)
in \cref{eq:6} should therefore exclude small \(m\).

Note also that even in the \(\dot{n}=0\) case \cref{eq:10} does not
vanish. The coupling of the modes originates from the (spatially
modulated) deviation of \(n\) from \(n_{0}\). It is also clear, that
the choice of \(n_{1}(ϕ,t)\) has to be made so that
\(\int_{0}^{1}\abs{n_{0}^{2}/n_{1}^{2}-1}\dd{ϕ}\) is minimized to best
approximate the condition \(δ\ll 1\) by minimizing \(κ_{lm}\) and
justify the definition of the \(ω_{m}\). Remember that the \(ω_{m}\)
are still free parameters and have to be chosen so that
\({∂_{t}{b}_{m}}/{b_{m}}\sim δ \ll 1\).  In particular
\(n_{1}(ϕ, t) = \mathrm{const}\) is not a valid choice. Preferably,
one should define the \(ω_{m}\) to minimize the \(κ_{lm}\) in
\cref{eq:10}. This also yields the exact solution in the
\(n(ϕ,t)=\mathrm{const}\) case.

For time independent \(n_{1}\) we can find suitable \(ω_{m}\) by
minimizing
\begin{equation}
  \label{eq:15}
  \bqty{∫_{0}^{2π}\pqty{\frac{m^{2}c^{2}}{n^{2}R^{2}} - ω_{m}^{2}}\dd{ϕ}}^{2}
\end{equation}
giving
\begin{equation}
  \label{eq:14}
  ω_{m}^{2}= \frac{1}{2π} \frac{m^{2}c^{2}}{R^{2}} ∫_{0}^{2π}
  n(ϕ)^{-2}\dd{ϕ}=\pqty{\frac{m c}{R n_{\mathrm{eff}}}}^{2},
\end{equation}
where \(n_{eff}^{-2}=(2π)^{-1}∫_{0}^{2π}n^{-2}\dd{ϕ}\). If \(n_{1}\)
depends on time, one may use the long time average of \cref{eq:14}.


\subsection{Modulation of a small portion of the ring.}
\label{sec:modul-small-port}

We now turn to the case of the modulation of a small angular portion
\(ϕ_{W}\) of the ring. In such a case
\begin{equation}
  \label{eq:5}
  n=n_{0}+n_{1}(t)\rect\pqty{\frac{ϕ}{ϕ_{W}}},
\end{equation}
where \(\rect(x)=Θ(1/4-x^{2})\) and \(Θ\) is the Heaviside step
function. As the tangential components of the electric field are
continuous, we face no problems on this front.

If we further demand \(\abs{\max_{t} n_{1}(t)}\ll 1\) and \(\lim_{T\to
∞} T^{-1} ∫_{0}^{T}n_{1}(t)\dd{t} = 0\) we can choose
the \(ω_{n}\) as in \cref{eq:8}, as follows from \cref{eq:12}
\begin{equation}
  \label{eq:16}
  ω_{m}^{2}=\lim_{T\to
    ∞}\frac{1}{T}∫_{0}^{T}\frac{m^{2}c^{2}}{R^{2}}\bqty{\frac{1}{n_{0}^{2}}+ϕ_{W}\pqty{\frac{1}{n^{2}}-\frac{1}{n_{0}^{2}}}}\dd{t}
  \approx \lim_{T\to
    ∞}\frac{1}{T}∫_{0}^{T}\frac{m^{2}c^{2}}{R^{2}}\bqty{\frac{1}{n_{0}^{2}}
    - 2ϕ_{W}\pqty{\frac{n_{1}(t)}{n_{0}}}}\dd{t} =\pqty{\frac{mc}{Rn_{0}}}^{2}.
\end{equation}

To connect to the result in \cite{Dutt2019}, we can then further
evaluate \cref{eq:10}, using
\(C_{m}=\sqrt{\frac{\hbar \abs{ω_{m}}}{4π R ε_{0}n_{0}^{2}}}\) to find
\begin{equation}
  \label{eq:17}
  κ_{lm}=\frac{Ω_{R}ϕ_{W}\abs{l}^{-\frac{3}{2}}\abs{m}^{\frac{5}{2}}}{4π}
  \pqty{\frac{n_{0}^{2}}{(n_{0}+n_{1}(t))^{2}}-1}\sinc\pqty{(m-l)\frac{ϕ_{W}}{2}}\sgn(l).
\end{equation}

Note that here, we introduced an additional sign compared to
\cite{Dutt2019}, and we already transformed to a rotating frame.

A slightly different choice of normalization
\begin{equation}
  \label{eq:13}
  C_{m}=\frac{1}{m^{2}}\sqrt{\frac{\hbar \abs{ω_{m}}}{4π R
      ε_{0}n_{0}^{2}}}
\end{equation}
makes \(κ_{lm}\) hermitian and reproduces the result of
\cite{Dutt2019}
\begin{equation}
  \label{eq:18}
  \begin{aligned}
    κ_{lm}&=\frac{Ω_{R}ϕ_{W}\sqrt{ml}}{4π}
            \pqty{\frac{n_{0}^{2}}{(n_{0}+n_{1}(t))^{2}}-1}\sinc\pqty{(m-l)\frac{ϕ_{W}}{2}}\sgn(l)\\
    &\approx
    \frac{Ω_{R}ϕ_{W}\sqrt{ml}}{2π}
    \frac{n_{1}(t)}{n_{0}}\sinc\pqty{(m-l)\frac{ϕ_{W}}{2}}\sgn(l).
  \end{aligned}
\end{equation}

As we are interested in the case where \(m,l\gg 1\) and
\(m=M+δm\), \(l=M+δl\) with \(δm,δl\ll M\), the pre-factor of \(κ_{lm}\)
can be regarded as constant either way, guaranteeing hermiticity.

Note that compared to \cite{Dutt2019} we added an additional sign in
\cref{eq:9} and swapped the index \(m,l\) with \(-m,-l\). This leads to
\cref{eq:18} having the same sign as in this publication, rather than
the opposite as one would expect from \cref{eq:9} alone.

This constitutes a rigorous derivation of the result in that paper.


Evaluating \cref{eq:11} yields
\fixme{to be done}.

\subsection{Open Questions}
\label{sec:open-questions}

It would be interesting to compute an explicit expression for the
magnetic field as well. To do so would allow us to check the
continuity for \(μ\vb{H}=\vb{B}\) and to compute the Poynting vector
and give the modes a propagation direction. Of course this direction
can be inferred from the \(n=n_{0}=\mathrm{const}\) case.

It would also be of interest, to find out how good the paraxial
approximation captures the real field at the center of the fiber.

Lastly, the case of time independent \(n\) can likely be solved
exactly fairly easily. It would be interesting to see how this
solution relates to the equation found here.

\section{Engineering Hamiltonians}
\label{sec:engin-hamilt}

Having obtained the basic equations of motion in
\cref{sec:equat-moti-modul}, we now proceed to explore how to engineer
model Hamiltonians out of this equation.

\subsection{Notation and Preliminaries}
\label{sec:notat-prel}


Before we begin to detail procedures to obtain engineered
Hamiltonians, a few notions concerning notation and the general
assumptions will be introduced.

On the physical level, we work with Fourier components \(b_{m}(t)
\eu^{\iu (mϕ-ω_{m}t)}\) of
the electric Field in the ring resonators in an appropriate frame.
The amplitudes \(b_{m}\) can then be identified with a quantum state
by defining
\begin{equation}
  \label{eq:19}
  \ket{ψ} \equiv ∑_{m} b_{m}\ket{m}
\end{equation}
with \(\ket{m}\) being orthonormal unit vectors
(\(\braket{m}{n}=δ_{mn}\)) in a Hilbert space \(\hilb\). This defines
the fiducial basis in which the state can be measured by recording the
slowly changing envelopes of the modes.

In this language \cref{eq:9} (in the non-rotating frame) becomes
\begin{equation}
  \label{eq:20}
  \iu ∂_{t}\ket{ψ} = H(t) \ket{ψ},
\end{equation}
with \(H_{mn}(t)= κ_{mn}(t)\eu^{-\iu{ω_{n}-ω_{m}}t}\) where we've neglected the \(γ_{mn}\)
contribution from \cref{eq:11}. Let us also define
\begin{equation}
  \label{eq:21}
  \begin{aligned}
    [D(ω)]_{mn} &\equiv ω_{m} δ_{mn} & κ_{mn}(t) & \equiv Δ_{mn}
                                                   \frac{n_{1}(t)}{n_{0}} \equiv Δ_{mn}\, f(t),
  \end{aligned}
\end{equation}
where \(n(t) = n_{0} + n_{1}(t)\) with \(n_{1}\ll n_{0}\) as discussed
in \cref{sec:modul-small-port}.

The above may seem rather obvious, but a clarification of conventions
is crucial.
In that vein the Hamiltonian of \cref{eq:20} can be expressed as
\begin{equation}
  \label{eq:22}
  H = H_{0} + V(t)
\end{equation}
with \(H_{0}=D(ω)\) and \(V(t)=Δ\, f(t)\).

For practical reasons and in order for the assumptions of
\cref{sec:pert-maxw-equat} to hold, we will always work with finite
set of resonator modes, making our effective Hamiltonians finite
dimensional. Further, it is assumed that we stimulate and modulate the
system in such a way, that the boundary conditions in mode space are
not important, so that they can be chosen at our convenience.  This
means that we're only concerned with a finite subspace
\(\hilb_{\mathrm{phys}} = \qty{\ket{m} \colon m\in
  \bqty{{m_{0}-(N-1)/2},{m_{0} + (N-1)/2}}} \subset \hilb\).

The goal is now to choose the geometry and the modulation so that
the time evolution operator
\begin{equation}
  \label{eq:23}
  \mathcal{U}_{t}[H] = \mathcal{T}\,\exp(-\iu ∫_{0}^{t}H(t))
\end{equation}
for the Hamiltonian \(H\) of \cref{eq:20} matches the time evolution
operator for some reference Hamiltonian \(H_{\target}\)
\begin{equation}
  \label{eq:24}
  \norm{\mathcal{U}_{t}[U H U^\dag]-\mathcal{U}_{t}[H_{\target}]} < ε
  \iff \norm{\bqty{U_{t}[U H U^†]-U_{t}[H_{\target}]}\ket{ψ}} \leq ε.
\end{equation}
for \(0\leq t\leq T\), where \(\norm{\cdot}\) on the left side is the
operator norm restricted \(\hilb_{\mathrm{phys}}\), \(U\) is some
unitary and \(ε>0\). The unitary \(U\) is allowing for a basis change
relative to the physical basis \cref{eq:19}. We require unitary
equivalence with the same unitary for all times. As
\(\mathcal{U}_{t}[H]\) is invertible, it would be trivial to achieve
perfect equivalence using a time dependent unitary
\begin{equation}
  \label{eq:25}
  U(t)=\mathcal{U}_{t}[H_{\target}] \mathcal{U}_{t}[H_{\target}]^†.
\end{equation}

As transformations into rotating frames are necessary we have to
loosen the requirement and allow those transformations as well. These
transformations then amount to considering oscillatory linear
combinations of the oscillator mode amplitudes.

This ``problem'' does not occur in \cite{Dutt2019}, as there the
response to a certain input is measured, yielding a band
structure. This requires the actual Floquet energies \emph{in the
  rotating frame} to match the eigenenergies of the target
Hamiltonian. But if observables such as the mean displacement of a
walker such as in \refcite{Ricottone2020} are to be computed, the
measured quantity should be the amplitudes, or equivalently the state
\(\ket{ψ}\). These should be obtained directly using only simple
transformations such linear combinations (constant unitaries) and
translation of the signals in frequency space (rotating frame). For if
we knew the detailed dynamics induced \(H_{target}\) already, there
would be little point in trying to simulate them in the first place
and using an elaborate transformation such as \cref{eq:25} would
defeat the point.


It is useful to express the above in terms of an effective Hamiltonian
\begin{equation}
  \label{eq:26}
  H_{\eff}[H](t)\equiv \frac{1}{-\iu t} \log[\mathcal{U}_{t}[H]].
\end{equation}
This Hamiltonian, similar to the Floquet Hamiltonian, has a spectrum
limited by the branch cut of the complex logarithm, which however has
no influence on the dynamics it generates. By continuity of the
operator exponential the closeness
\(\norm{H_{\eff}[H](t) -H_{\eff}[H_{\target}](t)}  \leq ε/t\) of the
effective Hamiltonians implies the condition \cref{eq:24}. This
representation lends itself particularly well to visualizations and
numerical studies.

If one is only interested in specific initial states, then one could
specify the less strict requirement, that the evolution of these
specific states should not deviate from the target.

Another possibility to loosen restrictions would be to only demand the
coincidence of the time evolution operators or the effective
Hamiltonians for a specific time \(t\).

\section{A Single Fiber Loop}
\label{sec:single-fiber-loop}
This section mostly follows~\refcite{Dutt2019}, focusing on how to
engineer a one-dimensional tight-binding Hamiltonians with one orbital
per unit cell,
\begin{equation}
  \label{eq:27}
  H = ∑_{m,n=-M}^{M}t_{mn} \ketbra{m_{0}+m}{m_{0}+n}.
\end{equation}

There we employ periodic boundary conditions to simplify the
calculations, so that \(\ket{m+N} = \ket{m}\) and
\(t_{m,n}=\min_{l\in \ZZ}{\abs{m-n + l N}}\). Here, we choose
\(N=2M +1\) for some \(M\in\NN\) and \(m_{0}\) so, that the relevant
subspace \(\hilb_{phys}\) is contained in it so that states within
this subspace don't ``see'' the boundary at the relevant time scales.


\section{Measuring the State Amplitudes}
\label{sec:meas-state-ampl}

TBD

\printbibliography{}
\end{document}


%%% Local Variables:
%%% mode: latex
%%% TeX-master: t
%%% TeX-output-dir: "output"
%%% TeX-engine: luatex
%%% End:
